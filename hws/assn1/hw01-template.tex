\documentclass[11pt]{article}
\usepackage{fullpage}
\usepackage{latexsym}
\usepackage{verbatim}
\usepackage{code,proof,amsthm,amssymb, amsmath}
\usepackage{ifthen}
\usepackage{graphics}
\usepackage{mathpartir}

%% Question macros
\newcounter{question}[section]
\newcounter{extracredit}[section]
\newcounter{totalPoints}
\setcounter{totalPoints}{0}
\newcommand{\question}[1]
{
\bigskip
\addtocounter{question}{1}
\addtocounter{totalPoints}{#1}
\noindent
{\textbf{Task \thesection.\thequestion}}[#1 points]:


}
\newcommand{\ecquestion}
{
\bigskip
\addtocounter{extracredit}{1}
\noindent
\textbf{Extra Credit \thesection.\theextracredit}:}
  
%% Set to "false" to generate the problem set, set to "true" to generate the solution set
\def\issolution{false}

\newcounter{taskcounter}
\newcounter{taskPercentCounter}
\newcounter{taskcounterSection}
\setcounter{taskcounter}{1}
\setcounter{taskPercentCounter}{0}
\setcounter{taskcounterSection}{\value{section}}
\newcommand{\mayresettaskcounter}{\ifthenelse{\value{taskcounterSection} < \value{section}}
{\setcounter{taskcounterSection}{\value{section}}\setcounter{taskcounter}{1}}
{}}

% Solution-only - uses an "input" so that it's still safe to publish the problem set file 
\definecolor{solutioncolor}{rgb}{0.0, 0.0, 0.5}
\newcommand{\solution}[1]
  {\ifthenelse{\equal{\issolution}{true}}
  {\begin{quote}
%    \addtocounter{taskcounter}{-1}
    \fbox{\textcolor{solutioncolor}{\bf Solution \arabic{section}.\arabic{taskcounter}}}
    \addtocounter{taskcounter}{1}
    \textcolor{solutioncolor}{\input{./solutions/#1}}
  \end{quote}}
  {}}

\newcommand{\qbox}{\fbox{???}}

\setcounter{taskcounter}{1}
\setcounter{taskPercentCounter}{0}
\setcounter{taskcounterSection}{\value{section}}

\newcommand{\ttt}[1]{\texttt{#1}}

%% part of a problem
\newcommand{\task}[1]
  {\bigskip \noindent {\bf Task\mayresettaskcounter{}\addtocounter{taskPercentCounter}{#1} \arabic{taskcounter}\addtocounter{taskcounter}{1}} (#1\%).}

\newcommand{\ectask} 
  {\bigskip \noindent {\bf Task\mayresettaskcounter{} \arabic{section}.\arabic{taskcounter}\addtocounter{taskcounter}{1}} (Extra Credit).}
  
\newcommand{\val}[1]{#1~\textsf{val}}
\newcommand{\num}[1]{\texttt{num}[#1]}
\newcommand{\str}[1]{\texttt{str}[#1]}
\newcommand{\plus}[2]{\texttt{plus}(#1; #2)}
\newcommand{\mult}[2]{\texttt{times}(#1; #2)}
\newcommand{\cat}[2]{\texttt{cat}(#1; #2)}
\newcommand{\len}[1]{\texttt{len}(#1)}
\newcommand{\abst}[2]{#1.#2}
\newcommand{\letbind}[3]{\texttt{let}(#1; \abst{#2}{#3})}
\newcommand{\steps}[2]{#1 \mapsto #2}
\newcommand{\subst}[3]{[#1/#2]#3}
\newcommand{\err}[1]{#1~\textsf{err}}

\newcommand{\proves}{\vdash}
\newcommand{\hasType}[2]{#1 : #2}
\newcommand{\typeJ}[3]{#1 \proves \hasType{#2}{#3}}
\newcommand{\ctx}{\Gamma}
\newcommand{\emptyCtx}{\emptyset}
\newcommand{\xCtx}[2]{\ctx, \hasType{#1}{#2}}
\newcommand{\typeJC}[2]{\typeJ{\ctx}{#1}{#2}}
\newcommand{\numt}{\texttt{num}}
\newcommand{\strt}{\texttt{str}}


\title{Assignment \#1: \\
        Abstract Binding Trees, Dynamics and Statics}

\author{15-312: Principles of Programming Languages}
\date{Out: Tuesday, January 23rd, 2013\\
      Due: Monday, February 4th, 2013 11:59pm}

\begin{document}
\maketitle
\subsection*{Abstract Binding Trees}

\task{5} Suppose \texttt{v} has type \texttt{t view}. Will \texttt{out(into(v))} be equal to \texttt{v}? Explain why or why not.

Think carefully about what effect the renaming \texttt{out} can
perform before answering this question.

\subsection*{Statics}
\addtocounter{taskcounter}{4}
\task{10} (Unicity of Typing) Prove that for every typing context $\ctx$ and expression $e$, there exists at most one $\tau$ such that $\typeJC{e}{\tau}$.

\task{10} (Canonical Forms) Prove that if $\val{e}$, then 
\begin{enumerate}
\item if $\typeJC{e}{\numt}$ then $e = \num{n}$ for some number $n$.
\item if $\typeJC{e}{\strt}$ then $e = \str{s}$ for some string $s$.
\end{enumerate}

%\newpage
%\appendix
%\section{Dynamics of \Lnumstr}
%\noindent
%$\fbox{\inferrule{}{\val{e}}}$
%\begin{mathpar}
%\inferrule{ }
%          {\val{\num{n}}}
%          
%\inferrule{ }
%		  {\val{\str{s}}}
%\end{mathpar}
%$\fbox{\inferrule{}{\steps{e}{e'}}}$
%\begin{mathpar}
%\inferrule{ }{
%	\steps{\plus{\num{n_1}}{\num{n_2}}}{\num{n_1 + n_2}}
%}
%
%\inferrule{
%	\steps{e_1}{e_1'}
%}{
%	\steps{\plus{e_1}{e_2}}
%		  {\plus{e_1'}{e_2}}
%}
%
%\inferrule{
%	\steps{e_2}{e_2'}
%}{
%	\steps{\plus{\num{n_1}}{e_2}}
%		  {\plus{\num{n_1}}{e_2'}}
%}
%
%\inferrule{ }{
%	\steps{\mult{\num{n_1}}{\num{n_2}}}{\num{n_1 * n_2}}
%}
%
%\inferrule{
%	\steps{e_1}{e_1'}
%}{
%	\steps{\mult{e_1}{e_2}}
%		  {\mult{e_1'}{e_2}}
%}
%
%\inferrule{
%	\steps{e_2}{e_2'}
%}{
%	\steps{\mult{\num{n_1}}{e_2}}
%		  {\mult{\num{n_1}}{e_2'}}
%}
%
%\inferrule{ }{
%	\steps{\cat{\str{s_1}}{\str{s_2}}}{\str{s_1 \text{\textasciicircum} s_2}}
%}
%
%\inferrule{
%	\steps{e_1}{e_1'}
%}{
%	\steps{\cat{e_1}{e_2}}
%		  {\cat{e_1'}{e_2}}
%}
%
%\inferrule{
%	\steps{e_2}{e_2'}
%}{
%	\steps{\cat{\str{s_1}}{e_2}}
%		  {\cat{\str{s_1}}{e_2'}}
%}
%
%\inferrule{ }{
%	\steps{\len{\str{s}}}{\num{|s|}}
%}
%
%\inferrule{
%	\steps{e}{e'}
%}{
%	\steps{\len{e}}{\len{e'}}
%}
%
%\inferrule{\val{e_1}}{
%	\steps{\letbind{e_1}{x}{e_2}}{\subst{e_1}{x}{e_2}}
%}
%
%\inferrule{
%	\steps{e_1}{e_1'}
%}{
%	\steps{\letbind{e_1}{x}{e_2}}
%		  {\letbind{e_1'}{x}{e_2}}
%}
%\end{mathpar}
%$\fbox{\inferrule{}{\err{e}}}$
%\begin{mathpar}
%\inferrule{ }{
%	\err{\plus{\str{s}}{e_2}}
%}
%
%\inferrule{ }{
%	\err{\plus{\num{n}}{\str{s}}}
%}
%
%\inferrule{
%	\err{e_1}
%}{
%	\err{\plus{e_1}{e_2}}
%}
%
%\inferrule{
%	\err{e_2}
%}{
%	\err{\plus{\num{n}}{e_2}}
%}
%
%\inferrule{ }{
%	\err{\mult{\str{s}}{e_2}}
%}
%
%\inferrule{ }{
%	\err{\mult{\num{n}}{\str{s}}}
%}
%
%\inferrule{
%	\err{e_1}
%}{
%	\err{\mult{e_1}{e_2}}
%}
%
%\inferrule{
%	\err{e_2}
%}{
%	\err{\mult{\num{n}}{e_2}}
%}
%
%\inferrule{ }{
%	\err{\cat{\num{n}}{e_2}}
%}
%
%\inferrule{ }{
%	\err{\cat{\str{s}}{\num{n}}}
%}
%
%\inferrule{
%	\err{e_1}
%}{
%	\err{\cat{e_1}{e_2}}
%}
%
%\inferrule{
%	\err{e_2}
%}{
%	\err{\cat{\str{s}}{e_2}}
%}
%
%\inferrule{ }{
%	\err{\len{\num{n}}}
%}
%
%\inferrule{\err{e}}{
%	\err{\len{e}}
%}
%
%\inferrule{\err{e_1}}{
%	\err{\letbind{e_1}{x}{e_2}}
%}
%\end{mathpar}
%
%\section{Statics of \Lnumstr}
%$\fbox{\inferrule{}{\typeJC{e}{\tau}}}$
%\begin{mathpar}
%\inferrule{\hasType{x}{\tau} \in \ctx}{
%	\typeJC{x}{\tau}
%}
%
%\inferrule{ }{
%	\typeJC{\num{n}}{\numt}
%}
%
%\inferrule{ }{
%	\typeJC{\str{s}}{\strt}
%}
%
%\inferrule{
%	\typeJC{e_1}{\numt}\\
%	\typeJC{e_2}{\numt}
%}{
%	\typeJC{\plus{e_1}{e_2}}{\numt}
%}
%
%\inferrule{
%	\typeJC{e_1}{\numt}\\
%	\typeJC{e_2}{\numt}
%}{
%	\typeJC{\mult{e_1}{e_2}}{\numt}
%}
%
%\inferrule{
%	\typeJC{e_1}{\strt}\\
%	\typeJC{e_2}{\strt}
%}{
%	\typeJC{\cat{e_1}{e_2}}{\strt}
%}
%
%\inferrule{
%	\typeJC{e}{\strt}
%}{
%	\typeJC{\len{e}}{\numt}
%}
%
%\inferrule{
%	\typeJC{e_1}{\tau_1}\\
%	\typeJ{\xCtx{x}{\tau_1}}{e_2}{\tau_2}
%}{
%	\typeJC{\letbind{e_1}{x}{e_2}}{\tau_2}
%}
%\end{mathpar}
%

\end{document}

